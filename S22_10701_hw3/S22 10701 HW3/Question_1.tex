\section*{Problem 1: K-Nearest Neighbors - Black Box [10 Points]}

\begin{enumerate}

        \item \textbf{[6 pts]} In a KNN classification problem, assume that the distance measure is not explicitly specified to you. Instead, you are given a “black box” where you input a set of instances $P_1, P_2, \dots P_n$ and a new example $Q$, and the black box outputs the nearest neighbor of $Q$, say $P_i$ and its corresponding class label $C_i$. Is it possible to construct a KNN classification algorithm (w.r.t the unknown distance metrics) based on this black box alone? If so, how and if not, why not?
        \begin{tcolorbox}[fit,height=7cm, width=0.9\textwidth, blank, borderline={1pt}{-2pt}]
        %solution 
        \end{tcolorbox}
        
        \item \textbf{[4 pts]} If the black box returns the $j$ nearest neighbors (and their corresponding class labels) instead of the single most nearest neighbor (assume $j \neq k$), is it possible to construct a KNN classification algorithm based on the black box? If so how, and if not why not?
        \begin{tcolorbox}[fit,height=7cm, width=0.9\textwidth, blank, borderline={1pt}{-2pt}]
        %solution 
        \end{tcolorbox}
    

    \clearpage
\end{enumerate}